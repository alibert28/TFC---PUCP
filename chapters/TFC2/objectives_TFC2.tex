\chapter*{\MakeUppercase{Planeación y Metodología para TFC2}}
\thispagestyle{mainmatterstyle} % Cambia el estilo para que este numerado en la esquina superior derecha. Esta presente en todos los chapter
\addcontentsline{toc}{chapter}{\MakeUppercase{Planeación y Metodología para TFC2}}

\section{Objetivos para de TFC2}
Se establece un objetivo general para el curso TFC2, del cual se derivan objetivos específicos, enumerados en \ref{lst:objetivos_especificos_TFC2}, diseñados para garantizar la consecución del objetivo principal a lo largo del desarrollo del curso.

\subsection{Objetivo general}

El objetivo principal para el curso TFC2 es elaborar los diseños mecánicos, eléctricos, electrónicos, de sistemas de control y de software, además de calcular los costos asociados a estos diseños. Esta labor se emprende con el con el fin de cumplir con el objetivo general del trabajo de investigación.

\subsection{Objetivos específicos}

\begin{enumerate}
	\setlength\itemsep{-0.5em}
	\item Diseñar componentes mecánicos detallados que cumplan con los requisitos funcionales y de seguridad establecidos, utilizando herramientas de diseño asistido por computadora (CAD) para su modelado y simulación.
	\item Crear esquemas eléctrico y electrónicos precisos para la implementación del proyecto, incluyendo la selección de componentes y la configuración de circuitos, garantizando la compatibilidad y funcionalidad de los sistemas.
	\item Diseñar e implementar sistemas de control eficientes que optimicen la operación y desempeño del proyecto, incluyendo la programación del hardware (SBC) y sistemas embebidos.
	\item Desarrollar software personalizado para la interfaz de usuario, control, y procesamiento de datos, asegurando su integración fluida con los componentes mecánicos y electrónicos, y su usabilidad por parte de los usuarios finales.
	\item Estimar de manera precisa los costos asociados con el desarrollo de los diseños mecánicos, eléctricos/electrónicos, de sistemas de control y de software, incluyendo costos de materiales, manufactura, y operación.
	\label{lst:objetivos_especificos_TFC2}
\end{enumerate}

\section{Metodología para alcanzar los objetivos planteados para TFC2}

Para alcanzar los objetivos específicos del curso TFC2, se propone el avance en el desarrollo del concepto de solución óptima seleccionado durante el curso TFC1. Este proceso comenzará con el diseño del sistema mecánico en el Capítulo \ref{me_design}, donde se definirán las especificaciones dimensionales de acuerdo a los requisitos del sistema y se realizará una selección cuidadosa de materiales para cada componente. Para validar la integridad mecánica, se emplearán simulaciones de Elementos Finitos (FEA) para confirmar la rigidez y resistencia del sistema, así como simulaciones de Sistemas de Múltiples Cuerpos (MBS) para asegurar su adecuada posición espacial.

A continuación, en el Capítulo \ref{ee_design}, se procederá con el diseño del sistema eléctrico/electrónico, que incluirá el desarrollo de esquemáticos electrónicos y un tablero eléctrico para garantizar un suministro de energía seguro al sistema. Se seleccionarán y programarán los componentes hardware necesarios.

En el Capítulo \ref{co_design}, se diseñará y ajustará el algoritmo de control de acuerdo con los requisitos de tiempo de funcionamiento del sistema. El Capítulo \ref{so_design} se dedicará al desarrollo de una interfaz de usuario que integre Redes Neuronales Artificiales (RNA) para el procesamiento de imágenes requerido por el sistema.

Finalmente, en el Capítulo \ref{ct_design}, se llevará a cabo una estimación de los costos asociados mediante el análisis de precios de componentes, costos de manufactura y operación. El proyecto concluirá con ciclos de pruebas y ajustes para asegurar el cumplimiento de los objetivos establecidos.

Para todos los diseños, se elaborarán planos detallados que incluyan dimensiones y tolerancias necesarias para la fabricación. En el caso de los diseños eléctricos/electrónicos, se especificará la lista de componentes de cada subcircuito del sistema.
