\section{Metodología}
% BENDEZU_LEDESMA_JORGE_DISEÑO_PROTOTIPO_DRON

Para el desarrollo del presente trabajo se emplearan adaptaciones de las metodologías diseñadas por la VDI (Asociación Alemana de Ingenieros). Entre ellas, se utilizar una adaptación de la metodología VDI 2206 \cite{VDIVDE2206_2021} y de la metodología VDI 2221 \cite{VDI2221_2019} para el diseño conceptual, el cual consistirá en las siguientes etapas: Elaboración de lista de requerimientos, Elaboración de un diagrama de funciones y Elaboración de una matriz morfológica (con 3 soluciones). Posteriormente se seguirá una adaptación de la metodología VDI 2225 \cite{VDI2225_series} para realizar un análisis técnico y económica de las soluciones.

De ese modo, los Capítulos 1 y 2 se enfocaran en a comprender la problemática de la investigación. El Capítulo 1 aborda una investigación preliminar para contextualizar el problema, identificar los enfoques de la problemática y definir el alcance de la solución. El Capítulo 2, por su parte, examina el estado del arte de la tecnología para establecer un marco de referencia que permita evaluar la viabilidad del proyecto, con el fin de determinar parámetros y características relevantes para el desarrollo del sistema.

En el Capítulo 3, se desarrollan los procesos de la metodología correspondientes para diseñar la estructura de funciones y el concepto de solución. Utilizando la lista de requerimientos, se abstrae el problema para obtener una comprensión amplia de las funciones necesarias, lo que guía el diseño mecánico y eléctrico inicial y ayuda a determinar los sensores, actuadores y fuentes de energía necesarios. A partir de esta estructura, se crea una matriz morfológica que facilita la combinación de distintas alternativas de solución, culminando en la definición precisa de los objetivos de diseño e implementación. Además, se realizará una ponderación de los conceptos de solución a través de un análisis técnico-económico con el objetivo de obtener un concepto de solución óptimo.


Por último, en los capítulos subsecuentes del presente trabajo de investigación, se desarrollaran las siguientes partes del trabajo:
\begin{itemize}
	\setlength\itemsep{-0.5em}
	\item Realización de cálculos mecánicos, electrónicos y de control necesarios para determinar las características de la máquina.
	\item Se va a seleccionar (o diseñar) un modelo de inteligencia artificial para realizar la labor de la visión artificial del sistema, además, se va a buscar, o en su defecto, se va a crear un dataset para entrenar un modelo de inteligencia artificial.
	\item Verificación de las partes mecánicas por medio de software de elementos finitos.
	\item Elaboración de planos y estimación de costos de fabricación.
\end{itemize}

\section{Alcance de la Investigación}

Este trabajo se enfoca en el diseño de un sistema a través del dimensionamiento detallado de cada uno de sus componentes. Se debe señalar que el diseño del sistema esta orientado a un entorno no industrial y que busca alcanzar un Nivel 4 de Madurez Tecnológica (TRL4), mediante simulaciones que se llevarán a cabo tanto en software como a través de un Producto Mínimo Viable (MVP) en un entorno de laboratorio controlado. Para concluir, esta tesis presentará todos los cálculos de diseño necesarios, así como los planos mecánicos y electrónicos, los materiales escogidos para el proyecto, y los diagramas de flujo que explican la operación y control del sistema.