% ################################### PAQUETES ###################################
%% Codificación y Fuente
\usepackage[utf8]{inputenc} % Codificación de entrada en UTF-8
\usepackage[T1]{fontenc} % Codificación de fuente
\usepackage{times} % Cambia la fuente a Times New Roman
\pdfminorversion=7 % Por un warning por el logo en la protada

%% Internacionalización
\usepackage[spanish,es-lcroman]{babel} % spanish,es-lcroman son para cambiar la numeración romana a mínuscual. (Por defecto, solo spanish esta en mayuscula)

%% Geometría y Márgenes
%\PassOptionsToPackage{showframe}{geometry} % Habilitar la opción de mostrar margenes en el paquete "geometry"
\usepackage[left=2.54cm, right=2.54cm, top=2.54cm, bottom=2.54cm]{geometry} % Define los márgenes de la página

%% Matemáticas
\usepackage{amsmath} % Facilita la escritura de estructuras matemáticas
\usepackage{amssymb} % Proporciona una extensa colección de símbolos matemáticos, incluye \blacklozenge

%% Gráficos y Figuras
\usepackage{graphicx} % Permite incluir gráficos
\graphicspath{ {./img/} } % Ruta por defector para las imágenes
%\usepackage{subfigure} % Permite manejar subfiguras dentro de una figura
%\usepackage{wrapfig} % Permite que el texto rodee a las figuras

%% Tablas
%\usepackage{tabularx} % Tablas con anchuras ajustables
\usepackage{colortbl} % Agrega color a las tablas, incluye \cellcolor
\usepackage{longtable} % Para tablas que se extienden por varias páginas, incluye {longtable}
\usepackage{multirow} % Permite celdas que abarcan varias filas en tablas 
\usepackage{bigstrut} % Permite crear espacios más grandes en las tablas (incluye \bigstrut)
\usepackage{tabularray} % Paquete para crear tablas, incluye {longtblr}
\usepackage{array} % Extiende las opciones de las tablas
\usepackage{ltablex} % longtable con opciones de tabularx

%% Encabezados y Pies de Página
\usepackage{fancyhdr} % Permite personalizar encabezados y pies de página

%% Otros
\usepackage{xcolor} % Permite el uso de colores (cyan, yellow)
\usepackage{hyperref} % Gestiona enlaces hipertexto, incluye (\url)
\usepackage{rotating} % Permite la rotación de imagenes y tablas, incluye el entorno {sidewaysfigure}
\usepackage{scalerel} % Escala objetos a tamaños relativos, incluye \scaleobj
\usepackage{float} % Permitede finir objetos flotantes usando [H]
\usepackage[export]{adjustbox} % Permite ajustar objetos en entornos flotantes.
\usepackage{titletoc}% Personalización detallada de los contenidos, incluye \titlecontents
\usepackage[font=footnotesize]{caption} % Cambia el tamaño de las captions a 10pt (ya que el documento esta definido en 12pt)
\usepackage{titlesec} % Permite cambiar el tamaño, color, fuente de los títulos de chapter, section y subsection
%\usepackage{color} % Permite el uso de colores
%\usepackage{verbatim} % Mejora el entorno verbatim
%\usepackage{epstopdf} % Convierte imágenes EPS a PDF
%\usepackage{ifpdf} % Proporciona condicionales para la compilación PDF
%\usepackage{url} % Facilita la escritura de direcciones web
%\usepackage{vmargin} % Permite ajustar los márgenes de la página
%\usepackage{scrextend} % Añade funcionalidades de KOMA-Script
%\usepackage{anyfontsize} % Permite cualquier tamaño de fuente
%\usepackage{enumerate} % Personaliza las listas enumeradas
%\usepackage{indentfirst} % Indenta el primer párrafo de cada sección
%\usepackage{pdflscape} % Permite páginas individuales en formato paisaje
%\usepackage{flafter} % Asegura que los flotantes no aparezcan antes de su definición en el texto
%\usepackage[framemethod=tikz]{mdframed} % Crea cajas con marcos para destacar texto
%\usepackage{setspace} % Permite cambiar el interlineado
%\usepackage{comment} % Permite excluir bloques de texto del documento final
%\usepackage[overload]{textcase} % Cambia el modo de las mayúsculas

%% Algoritmos y Pseudocódigo
%\usepackage{algorithmicx,algpseudocode} % Facilita la escritura de algoritmos y pseudocódigo
%\usepackage{algorithm} % Entorno flotante para algoritmos

%Situacionales
\usepackage{lipsum} % Genera texto de relleno

% ############################ DEFINICIÓN DE NUEVOS COMANDOS ############################

% ------------------- CONFIGURACIÓN DE CAPTION DE FIGURAS Y TABLAS -------------------
\captionsetup[table]{width=.8\linewidth}

% -------------------------- CONFIGURACIÓN DEL INTERLINEADO ---------------------------
\renewcommand{\baselinestretch}{1.5} % Interlineado

% --------------------------- CONFIGURAR FOOTERS Y HEADERS ---------------------------
% Estilo para el frontmatter
\fancypagestyle{frontmatterstyle}{
	\fancyhf{} % Limpia el header and footer
	\fancyfoot[C]{\thepage} % En el centro del footer coloca la numeración
	\renewcommand{\headrulewidth}{0pt} % La línea se setea a 0
}

% Estilo para el mainmatter
\fancypagestyle{mainmatterstyle}{
	\fancyhf{} % Limpia el header and footer
	\fancyhead[R]{\thepage} % En la esquina superior derecha del header coloca la numeración
	\renewcommand{\headrulewidth}{0pt} % La línea se setea a 0
}

% ------------------------ CONFIGURACIÓN DE LOS HIPERVÍNCULOS ------------------------
\urlstyle{same} %Hacer que la fuente del URL de la bibliografía sea la misma
\hypersetup{ % Configura las opciones para los hiperenlaces en el documento
	colorlinks=true, % Activa el color de los enlaces en lugar de cajas de color
	citecolor=black, % Establece el color de los enlaces de las citas a negro
	filecolor=black, % Establece el color de los enlaces de archivos a negro
	linkcolor=black, % Establece el color de los enlaces internos (como secciones) a negro
	urlcolor=black % Establece el color de los enlaces de URL a negro
}

% ----------------- CONFIGURACIÓN DE FUENTES Y TAMAÑOS DE LOS TÍTULOS -----------------

% Configuración para el título del capítulo
\titleformat{\chapter}[display] % Tipo de visualización 'display' para capítulos
{\normalfont\normalsize\bfseries\centering} % Estilo del título del capítulo
{\MakeUppercase{\chaptertitlename} \thechapter} % Prefijo del título, ej. "Capítulo 1"
{12pt} % Espacio entre el prefijo y el título del capítulo
{\normalsize} % Tamaño del título del capítulo

% Configuración para el título de la sección
\titleformat{\section} % Afecta solo a las secciones
{\normalfont\normalsize\bfseries} % Estilo del título de la sección
{\thesection} % Prefijo del título, ej. "1.1"
{12pt} % Espacio entre el prefijo y el título de la sección
{} % Código adicional para antes del título, si es necesario

% Configuración para el título de la subsección
\titleformat{\subsection} % Afecta solo a las subsecciones
{\normalfont\normalsize\bfseries} % Estilo del título de la subsección
{\thesubsection} % Prefijo del título, ej. "1.1.1"
{12pt} % Espacio entre el prefijo y el título de la subsección
{} % Código adicional para antes del título, si es necesario

% ------------------------- CUADRADOS DE COLORES EN EL TEXTO --------------------------
% normal box
\newcommand{\sqboxs}{1.2ex}% the square size
\newcommand{\sqboxf}{0.6pt}% the border in \sqboxEmpty
\newcommand{\sqbox}[1]{\textcolor{#1}{\rule{\sqboxs}{\sqboxs}}}

% empty box
\newcommand{\sqboxEmpty}[1]{%
	\begingroup
	\setlength{\fboxrule}{\sqboxf}%
	\setlength{\fboxsep}{-\fboxrule}%
	\textcolor{#1}{\fbox{\rule{0pt}{\sqboxs}\rule{\sqboxs}{0pt}}}%
	\endgroup
}

% ----------------------------------- DEFINIR COLORES -----------------------------------
\definecolor{GoldenTainoi}{rgb}{1,0.815,0.4}
\definecolor{GreenYellow}{rgb}{0.486,1,0.239}

% ------------------------ AJUSTES PARA LAS TABLAS DE {longtblr} ------------------------
% Cambia el valor de contfoot-text
\DefTblrTemplate{conthead-text}{mymod}{(Continuación)}
\SetTblrTemplate{conthead-text}{mymod}
\DefTblrTemplate{contfoot-text}{mymod}{Continúa en la siguiente página}
\SetTblrTemplate{contfoot-text}{mymod}

% ------------------------------ NUEVOS TIPOS DE COLUMNA ------------------------------
\newcolumntype{C}[1]{>{\centering\arraybackslash}p{#1}} % Define un nuevo tipo de columna "C" que centra el contenido horizontalemente y tiene un argumento para modificar el ancho
\newcolumntype{M}[1]{>{\centering\arraybackslash}m{#1}} % Define un nuevo tipo de columna "M" que centra el contenido verticalmente y tiene un argumento para modificar el ancho

%\newcommand{\tabularxmulticolumncentered}[3]{%
%	\multicolumn{#1}{|>{\hsize=\dimexpr #1\hsize+2\arrayrulewidth+(#1-1)\tabcolsep\relax\centering\arraybackslash}X|}{#3}%
%}

%\newcommand{\custommulticol}[3]{%
%	\multicolumn{#1}{|>{\hsize=\dimexpr#1\hsize+#1\tabcolsep+\arrayrulewidth\relax}#2|}{#3}
%}
%	\multicolumn{#1}{|>{\hsize=\dimexpr1\hsize+1\tabcolsep+2\tabcolsep+2.2cm\relax}X|}{#1}


% ----------- COMANDO PARA INCLUIR IMAGEN CON MARGEN Y TAMAÑO PERSONALIZADO -----------
\newcommand{\IMM}[3]{
	\adjustbox{margin=#3}{\includegraphics[width=#1\textwidth]{#2}}
}
% Comando \IMM: Inserta una imagen con margen y tamaño personalizado.
% Parámetros:
%   #1: Ancho de la imagen como una fracción del ancho de texto (\textwidth).
%   #2: Ruta del archivo de la imagen.
%   #3: Margen alrededor de la imagen.

% ------------------------------ ESCALAMIENTO DE OBJETOS ------------------------------
\newcommand{\B}[2]{\scaleobj{#2}{#1}} % Comando para escalar objetos

% ------------------------------ DEFINIR CONSTANTES ------------------------------
\newcommand\anchoFPT{100pt} %Ancho de la columna de Funciones Parciales (Para dominios de Tres columnas)
\newcommand\anchoFP{70pt}
\newcommand\anchoLE{0.9\textwidth} % Ancho de Listas de exigencias
\newcommand\anchoLEdes{0.2\textwidth}
\newcommand\OpR{$\color{red}\B{\blacklozenge}{1.35}$}
\newcommand\OpA{$\color{blue}\B{\blacklozenge}{1.35}$}
\newcommand\OpV{$\color[HTML]{4CD514}{\B{\blacklozenge}{1.35}}$}
\newcommand\OpN{$\color{orange}\B{\blacklozenge}{1.35}$}

% --------------- PARA COLOCARLA LA PALABRA CAPÍTULO A CADA \chapter ---------------
%\titlecontents{chapter}% <section-type>
%[0pt]% <left>
%{\bfseries}% <above-code>
%{\chaptername\ \thecontentslabel:\quad}% <numbered-entry-format>
%{}% <numberless-entry-format>
%{\hfill\contentspage}% <filler-page-format>